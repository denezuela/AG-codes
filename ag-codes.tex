\documentclass{beamer}

\usepackage{ucs}
\usepackage[utf8]{inputenc}
\usepackage[english,russian]{babel}
\setbeamertemplate{navigation symbols}{}
\usepackage{ragged2e}
\justifying

\usepackage{commath}

\usetheme{CambridgeUS}
\usecolortheme{dolphin}

\setbeamertemplate {blocks} [rounded] [shadow=true]
\setbeamercolor {block title}{fg = white, bg = blue}

\begin{document}

\title{Исследование алгебро-геометрических кодов как кодов защиты от копирования}
\author[Загуменнов Д.В.]{Загуменнов Денис Владимирович}
\institute[ЮФУ]{Южный Федеральный университет

 Институт математики, механики и компьютерных наук им. И. И. Воровича

Кафедра алгебры и дискретной математики}
\date {04.12.2015}

\begin {frame}
\titlepage
\end {frame}

\section {Схемы специального широковещательного шифрования}

\begin {frame}
\tableofcontents[currentsection]
\end {frame}

\begin {frame}
\frametitle {ССШШ и их принципы}
ССШШ - схемы специального широковещательного шифрования
\newline
\begin{itemize}
            \item Свободное тиражирование данных в зашифрованном виде
            \item Уникальный набор ключей у легального пользователя
            \item Возможность коалиционной атаки
	 \item Списочный декодер + "хороший" код
\end{itemize}
\end {frame}

\section {Списочное декодирование}

\begin {frame}
\tableofcontents[currentsection]
\end {frame}

\begin {frame} {Списочное декодирование}
\begin{itemize}
	\item Возможность декодирования за пределами классического радиуса декодирования
            \item Результат в виде списка кодовых слов
            \item Кодовое слово, соответствующее исходному сообщению, находится в списке
\end{itemize}

\end {frame}

\section {c-TA коды}

\begin {frame}
\tableofcontents[currentsection]
\end {frame}

\begin {frame}  { c-TA коды }
\frametitle {Коалиции и её потомки}

Пусть {\sl C} --- код. Коалицией размера $c$ будем называть набор из ${\sl c}$ векторов кода, то есть
${\sl C_0} = \{ u^{(1)}, u^{(2)} , \dots , u^{(c)}\},  u^{(i)} \in {\sl C}$.
\newline

Потомками  коалиции ${\sl C_0}$ назовём множество
$$ {\rm desc}({\sl C_0})=\{(x_1, x_2, , \dots , x_n) \ | \  {\sl x_i} \ {\in} \ \{ u_i , u \ {\in} \
 {\sl C_0}\}\}.$$

\end {frame}

\begin {frame}
\frametitle {Определение c-TA}
Множество всех коалиций кода $C$ размера не больше $c$ обозначим ${\rm coal}_c(C)$.

Код $C$ называется $c-TA$ кодом, если выполнено следующее условие: $$ \forall \ v \in C \ \forall C_0 \in {\rm coal}_c(C) : v \in C \setminus C_0 $$
 $$ \forall \omega \in {\rm desc}(C_0) \  \exists y \in C_0 \to d(\omega, y) < d(v,y) $$
\end {frame}

\begin {frame}
\frametitle {Достаточное условие наличия у кода c-TA - свойства}
\begin {block} {Теорема. (A. Silverberg, J. Staddon, J. L. Walker. Applications of List Decoding to Tracing Traitors. Theorem 5)}
Пусть {\sl C} --- код, ${\sl n}$ --- длина кода {\sl C}, ${\sl d}$ --- минимальное кодовое расстояние кода {\sl C}.
 Если код {\sl C} удовлетворяет условию $ {\sl d} > {\sl n} - \frac {\sl n}{\sl c^2}$  $ ({\sl c} \ {\in} \  {\mathbb N}, {\sl c} \geq 2)$, то  код {\sl C} является {\sl с}-{\sl TA} кодом.

Причём, если ${\sl C_0} \in {\rm coal}_c(C)$   ${\sl w} \in {\rm desc(C_0)}$, то

1) $\exists \ {\sl x} \in {\sl C_0} : {\sl d}({\sl x}, {\sl w}) < {\sl n} - \frac {\sl n}{\sl c}$,

2) $ \forall \ {\sl x} \in {\sl C} : {\sl d}({\sl x}, {\sl w}) < {\sl n} - \frac {\sl n}{\sl c} \to {\sl x} \in {\sl C_0}$
\end {block}
\end {frame}

\begin {frame}
\frametitle {Код и декодер для ССШШ}
\begin {block}

Пусть $C$ --- код, $n$ --- его длина, $d$ --- минимальное кодовое расстояние. Пусть есть списочный декодер для $C$, исправляющий $r$ ошибок.

Для эффективного применения кода {\sl C} и списочного декодера для кода {\sl C} в ССШШ достаточно выполнения следующих условий:

1) $ {\sl d} > {\sl n} - \frac {\sl n}{\sl c^2}$

2)  $ {\sl r} > {\sl n} - \frac {\sl n}{\sl c}$
\end {block}
\end {frame}

\section {Алгебро-геометрические коды (L-конструкция)}

\begin {frame}
\tableofcontents[currentsection]
\end {frame}

\begin {frame}
Tom Høholdt, Jacobus H. van Lint and Ruud Pellikaan. Algebraic geometry codes.

Zhuo Zhia Dai. The Algebraic Geometric Coding Theory.

С.Г. Влэдуц, Д.Ю.Ногин, М.А.Цфасман. Алгеброгеометрические коды. Основные понятия.
\end {frame}

\begin {frame}  {Аффинное пространство}
Пусть ${\sl k}$ --- поле. Аффинное $n$-мерное пространство над полем ${\sl k}$ , точками которого являются наборы $$P =\{ x_1, x_2, \dots, x_n \}, x_i \in {\sl k},$$ будем обозначать ${\mathbb A^n}(k)$.
\end {frame}

\begin {frame}  {Аффинное многообразие}
Пусть ${\overline k}$ --- алгебраическое замыкание поля $k$.
\newline

Пусть $I \in {\overline k}[x_1, x_2, \dots, x_n]$ --- простой собственный идеал.
\newline
 \begin {block}

Аффинным многообразием называется множество: $$ X= X(I) = \{ P = (x_1, x_2, \dots, x_n) \in {\mathbb A^n}({\overline k}) : g(x_1, x_2, \dots, x_n) = 0 \forall g \in I \}.$$
\end {block}
Наборы $P$ называются точками многообразия.
\end {frame}


\begin {frame}  {Координатное кольцо и поле функций на многообразии}
 \begin {block}

Факторкольцо ${\overline k}[X] = {\overline k}[x_1, x_2, \dots, x_n] / I$ называется координатным кольцом многообразия $X$.
\newline

Поле частных координатного кольца называется полем функций многообразия $X$ и обозначается  ${\overline k}(X)$.
 \end {block}
\end {frame}

\begin {frame}  {Плоская аффинная кривая}
Пусть $f \in k[x,y]$ --- {\bf абсолютно неприводимый многочлен}, тогда $<f> = \{fh : h \in {\overline k}[x,y] \}$ --- простой идеал в  ${\overline k}[x_1, x_2].$
\newline

Многообразие $$C = C(f) = \{ P = (x, y) \in {\mathbb A^2}({\overline k}) : g(x,y) = 0 \  \forall  g \in <f> \}$$ называется плоской аффиной кривой.
\newline

Как нам вернуться к рассмотрению поля $k$?
\end {frame}

\begin {frame}
Рассматриваются только точки вида $$ P = (x, y) \in C : x, y \in k. $$
Такие точки называются рациональными.
Множество рациональных над полем $k$ точек на кривой $C$ обозначается $C(k)$.
\newline

Понятия кооррдинатного кольца и поля функций сужаются.

Координатным кольцом плоской аффиной кривой $C$ назовём  $$k[C] = k[x,y]/<f>,$$ а полем функций $k(C)$ --- поле частных $k[C]$.

\end {frame}

\begin {frame} {Пример}
$f = y - x^2, C = C(<f>), k = {\mathbb F}_2$

Точки (0,0), (1,1) --- рациональные точки, а ($\alpha, \alpha^2$) и ($\alpha^2, 1$) --- нерациональные точки.
\end{frame}

\begin {frame}  {Проективное пространство}
Проективное $n$-мерное пространство над ${\sl k}$, точками которого являются наборы вида $$Q = \{y_1 : y_2 : y_3 : \dots : y_{n+1}\} , y_i \in {\sl k},$$ где
\newline

1) не все $y_i$ равны нулю

2) наборы $ \{\lambda y_1 : \lambda y_2 : \lambda y_3 : \dots : \lambda y_n+1 \}, \lambda \in {\sl k}, \lambda \ne 0$ определяют одну и ту же точку
\newline

 будем обозначать ${\mathbb P^n}(k)$.
\end{frame}

\begin {frame}  {Плоская проективная кривая}
Рассматриваются только однородные многочлены из $k[X : Y : Z]$.
\newline

Понятие кривой и многообразия вводятся аналогично:

 абсолютно неприводимый однородный многочлен $\to$ простой однородный идеал $\to$ координатное кольцо $\to$ поле частных координатного кольца.
\newline

{\bf Замечание.} Поле функций на проективной кривой задаётся как подкольцо поля частных с однородными числителем и знаменателем одинаковой степени.
\newline
\end {frame}

\begin {frame}  {Плоская проективная кривая}
Любой неоднородный многочлен из $k[x,y]$ можно "проективизовать":  $f \in {\sl k}[x,y]$ --- неоднородный многолчен соответствует
\newline
 ${\sl F}({\sl X} : {\sl Y} : {\sl Z}) = {\sl Z^{\sl d}}{\sl f}({\frac {\sl X}{\sl Z}} , {\frac {\sl Y}{\sl Z}}) \in {\sl k}[ {\sl X} : {\sl Y} : {\sl Z}],$
 где ${\sl d} = {\rm deg}({\sl f})$.
\newline

Однородному многочлену из $k[X : Y : Z]$ поставим в соответствие многочлен из $k[x,y]$:
\newline
 ${\sl F}({\sl X} : {\sl Y} : {\sl Z}) = {\sl F}({\frac {\sl X}{\sl Z}} : {\frac {\sl Y}{\sl Z}} : 1)$, обозначим $x = \frac {X}{Z}, y = \frac {Y}{Z}$, тогда $F$ соответсвует $F(x,y) \in k[x,y].$
\end {frame}

\begin {frame} {Пример}
$f = y - x^2, C = C(<f>)$

$F = Z^2(Y/Z - X^2/Z^2) = YZ - X^2$

Рациональные точки над ${\mathbb F}_2$ --- точки (0 : 1 : 0), (1 : 1 : 1), (0 : 0 : 1).
\end {frame}

\begin {frame}  {Локальное кольцо точки}
Напомнание: координатным кольцом плоской аффиной кривой $C$ является $k[C] = k[x,y]/<f>,$ а полем функций $k(C)$ --- поле частных $k[C]$.
\newline

Будем считать $g, h \in k(C)$ одинаковыми элементами поля рациональных функций, если из $g$ обычными преобразованиями многочленов можно получить $h$, использую условие $f = 0$.
\newline

Пусть $\phi \in k(C), P \in C$. Говорят, что $\phi$ регулярна в точке $P$, если $$\exists g,h \in k(C) : \phi = \frac {g}{h}, h(P) \ne 0.$$
\end {frame}

\begin {frame}
$f = y - x^2, C = C(<f>)$
\newline

Функция $\frac {y}{x} = x = \frac{x}{1}$. Значит, $\frac {y}{x}$ регулярна в точке $(0,0)$.
\end {frame}

\begin {frame}  {Локальное кольцо точки}
\begin {block}

Пусть $P \in X$. Локальным кольцом точки $P$ называется кольцо ${\mathfrak O}_P$, состоящее из функций, регулярных в $P$, то есть: $$ {\mathfrak O}_P = \{ \phi \in k(C) : \exists \ g,h \in k(C) : \phi = \frac{g}{h}, h(P) \ne 0 \}.$$
\end {block}

\begin {block} {Теорема}

Локальное кольцо точки $P$ имеет единственный максимальный идеал ${\mathfrak M}_P$. Он состоит из функций, принимающих на точке $P$ значение 0: ${\mathfrak M}_P = \{\phi \in {\mathfrak O}_P : \phi (P) = 0\}$.
\end {block}
\end {frame}

\begin {frame}  {Неособые точки и гладкие кривые.}

\begin {block}

Точка $P$ называется неособой, если $\forall \phi \in k(C)  \phi \in {\mathfrak O}_P$ или $\phi^{-1} \in {\mathfrak O}_P$, в противном случае точка называется особой.
\end {block}

Аффинная кривая называется гладкой, если на ней нет особых точек. Далее будем рассматривать только гладкие кривые.
\end {frame}

\begin {frame} {Локальный параметр.}
\begin {block} {Теорема.}
Точка $P \in C$ неособа тогда и только тогда, когда максимальный идеал ${\mathfrak M}_P$ в локальном кольце точки $P$ -- главный,  то есть $\exists t \in {\mathfrak M}_P : m =\{ ta : a \in {\mathfrak O}_P\}.$
\end {block}

Если $P \in C$ -- неособая точка, то такая функция $t \in {\mathfrak M}_P$,

что ${\mathfrak M}_P = t{\mathfrak O}_P$, называется локальным параметром в точке $P$.
\end {frame}

\begin {frame} {Локальный параметр.}
\begin {block} {Теорема.}
Пусть $C$ -- гладкая кривая, $P \in C$, $t$ -- локальный параметр. Тогда любой элемент ${\mathfrak O}_P$ может быть представлен единственным образом в виде $ut^{n}, u \in  {\mathfrak O}_P \setminus {\mathfrak M}_P, n \in {\mathbb N}.$
\end {block}
\end {frame}

\begin {frame} {Дискретное нормирование.}
Любой элемент $\phi \in {\mathfrak O}_P$ может быть представлен единственным образом в виде $ut^{n}, u \in  {\mathfrak O}_P \setminus {\mathfrak M}_P, n \in {\mathbb N}.$

Рассмотрим функцию: $${\rm ord} : k(C) \to {\mathbb Z} \cup \infty,$$ заданную по правилу:

$ 1) \ {\rm ord}_P(0) = \infty \ \forall P \in C $

$ 2) \  {\rm ord}_P(\phi) = n, \phi \in {\mathfrak O}_P$

$ 3) \  {\rm ord}_P(\phi) = -n, \phi^{-1} \in {\mathfrak O}_P$.

%Функция обладает рядом свойств:

%${\rm ord}_P(gh) = {\rm ord}_P(g) + {\rm ord}_P(h)$

%${\rm ord}_P(g+h) \ge {\rm min}({\rm ord}_P(g), {\rm ord}_P(h))$.
\end {frame}

\begin {frame} {Дифференциальный признак.}
Пусть $f = \sum{a_{i,j}x^{i}y^{j}}$, тогда $f_x = \sum{a_{i,j}ix^{i-1}y^{j}}$,  $f_y = \sum{a_{i,j}jx^{i}y^{j-1}}$.

Аналогично, пусть $F = \sum{a_{i,j,k}X^{i}Y^{j}Z^{k}}$, тогда $F_X = \sum{a_{i,j,k}iX^{i-1}Y^{j}Z^{k}}$,  $F_Y = \sum{a_{i,j}jX^{i}Y^{j-1}Z^{k}}, F_Z = \sum{a_{i,j}kX^{i}Y^{j}Z^{k-1}} $.

\begin {block} {Теорема --- дифференциальный признак}

Пусть $C$ - аффинная кривая, $P=(a,b) \in C$, тогда если $f_y(P) \ne 0$, то $t=x-a$ является локальным параметром, а если $f_x(P) \ne 0$, то $t=y-b$ является локальным параметром. Если же $f_x = f_y = 0$, то $P$ является особой точкой.
\end {block}
\end {frame}

\begin {frame} {Дискретное нормирования поля функций на проективной кривой}
 Пусть $C$ -- плоская проективная кривая, заданная многочленом

 $F(X:Y:Z)$. Точка $P \in C$ называется особой, если $$F_X(P) = F_Y(P)=F_Z(P),$$ иначе называется неособой.
\newline

Пусть $P = (a : b : c)$ -- неособая точка на проективной кривой $C$, тогда:
$${\rm ord}_{(a : b : c)}(G(X : Y : Z)) = {\rm ord}_{(\frac {a}{c} :\frac {b}{c} : 1)}(G(\frac {X}{Z} : \frac {Y}{Z} : 1)) =  {\rm ord}_{(\frac {a}{c}, \frac {b}{c})}(G(x, y)).$$
\end {frame}

\begin {frame} {Дивизоры}
Пусть $С$ -- гладкая проективная кривая. Дивизором $D$ на $C$ называется формальная конечная сумма вида $D = \sum{a_PP}$, где $P$ -- точки на $C$, $a_P \in {\mathbb Z}$.
\newline

Если для дивизора $D = \sum{a_PP}$ все $a_P \ge 0$, то $D$ называют эффективным дивизором.
\newline

 Степенью дивизора ${\rm deg}D$ называется число $\sum{a_P}$.
\end {frame}

\begin {frame} {Главный дивизор функции}
 Пусть $С$ -- гладкая проективная кривая, $\phi \in k(C), \phi \ne 0$. Главным дивизором функции $\phi$ называется дивизор $$(\phi) = \sum_{P \in C}{{\rm ord}_P(\phi)P}.$$
\end {frame}

\begin {frame} {Пространство Римана-Роха.}
Пусть $D = \sum{a_PP}$ -- дивизор на гладкой проективной кривой $C$, пространством функций Римана-Роха, ассоциированного с дивизором $D$ называется $$L(D) = \{\phi \in k(C) \setminus \{0\} : (\phi) + D \ge 0\} \cup \{0\}.$$

Оно является векторным пространством над полем $k$.
\newline
\end {frame}

\begin {frame} {Род кривой}
\begin {block} {Формула Плюкера}
 Пусть $C$ -- гладкая {\bf плоская} проективная кривая, заданная многочленом $F$, пусть $deg(F)$ -- степень этого многочлена. Родом кривой $C$ будем называть число $$g = \frac{({\rm deg}(F)-1)({\rm deg}(F)-2)}{2}$$.
\end {block}
\end {frame}

\begin {frame} {Теорема Римана-Роха}
\begin {block} {Теорема}
Пусть $C$  -- гладкая проективная кривая рода $g$, тогда $\forall D$ --- дивизора на кривой $C$, такого, что $deg(D) \ge 2g - 1$ пространство Римана-Роха конечномерно, и $${\rm dim}(L(D)) = {\rm deg}(D) - g + 1$$.
\end {block}
\end {frame}

\begin {frame} {Алгеброгеометрический код (L-конструкция)}
Пусть $X$ - плоская гладкая кривая над произвольным полем, такая, что множество рациональных относительно поля Галуа ${\mathbb F_q}$ точек $X({\mathbb F_q})$ непусто.

Пусть ${\mathfrak P} \subset  X({\mathbb F_q}), {\mathfrak P} = \{P_1, \dots, P_n\}$, $|{\mathfrak P}| = n$, $D$ -- выбранный на $X$ дивизор.
\newline

Построим отображение $Ev_{\mathfrak P} : L(D) \to {\mathbb F_q}^n$ по правилу $$Ev(\phi)_{\mathfrak P} = \{ \phi(P_1), \dots, \phi(P_n) \}.$$
\end {frame}

\begin {frame} {Алгеброгеометрический код (L-конструкция)}
Получаем код $C = {\rm Im}(L(D)) \subset {\mathbb F_q}^n$, будем его обозначать $C.$

Дивизор $D$ будем называть дивизором кода $C$.
\newline

Пусть ${\rm dim}(L(D)) = m$. Если $\{\phi_1, \phi_2, \dots, \phi_ m\}$ -- базис в $L(D)$, то матрица
$$\begin{pmatrix}
\phi_1 (P_1) & \phi_1 (P_2) & \cdots & \phi_1 (P_n) \\
\phi_2 (P_1) & \phi_2 (P_2) & \cdots & \phi_2 (P_n) \\
\vdots & \vdots & \ddots & \vdots \\
\phi_ m (P_1) & \phi_ m (P_2) & \cdots &\phi_ m (P_n) \\
\end{pmatrix}$$

является порождающей матрицей кода.
\end {frame}

\begin {frame} {Алгеброгеометрический код (L-конструкция)}

\begin {block} {Теорема о параметрах кода}
Пусть $X$ -- кривая рода $g$, пусть $ 2g-1 \le {\rm deg}D = \alpha < n = |{\mathfrak P}|$. Тогда соответствующй алгеброгеометрический код код $C$ является $[n, k, d]_q$-кодом, где
$$k = \alpha - g + 1, d \ge d^* = n - \alpha.$$
\end {block}

Величина $d^* = n - \alpha$ называется конструктивным расстоянием кода. В дальнейшем рассматриваются только коды с конструктивным расстоянием.
\end {frame}

\begin {frame} {Пример}
$F = YZ - X^2, k = {\mathbb F}_7$.

$Q = (0 : 1 : 0) \in C$. Возьмём $D = mQ$.

Тогда получим алгеброгеометрический код с порождающей матрицей:
$$\begin{pmatrix}
\ 1 & \ 1 & \ 1 & \cdots & \ 1 \\
\ 0 & \ 1 & \ 2 & \cdots & 6 \\
\ 0 & \ 1 & \ 2^2 & \cdots & 6^2 \\
\vdots & \vdots & \ddots & \vdots \\
\ 0 & \ 1^m & \ 2^m & \cdots & 6^ m\\
\end{pmatrix}$$

\end {frame}

\section {Условия применения кодов и декодера}

\begin {frame}
\tableofcontents[currentsection]
\end {frame}

\begin {frame} {Условие c-TA}
\begin {block} {Теорема (A. Silverberg, J. Staddon, J. L. Walker. Applications of List Decoding to Tracing Traitors. Theorem 6)}
Пусть ${\sl n}$  - длина кода,  ${\sl k}$  - размерность кода, ${\sl g}$ - род кривой, на котором определён код, $D$ - дивизор кода $C$, ${\rm deg}(D) = \alpha \ge 2g - 1$.
Алгеброгеометрический код {\sl C} является {\sl с}-{\sl TA} кодом, если выполняется условие:
\begin {equation}
{\sl c} < \sqrt {\frac {\sl n}{k+g-1}}
\end {equation}
\end {block}
\end {frame}

\begin {frame} {Списочный декодер Судана-Гурусвами.}
\begin {block} {Теорема о радиусе работы}
Пусть {\sl C} - алгебро-геометрический код длины ${\sl n}$, $D$ - дивизор кода $C$, ${\rm deg}(D) = \alpha \ge 2g-1$. Тогда существует алгоритм декодирования (алгоритм декодирования Судана-Гурусвами) этого кода со сложностью, полиномиальной по ${\sl n}$, исправляющий не более ${\sl r}$ ошибок, где $r < {\sl n} - \sqrt { {\sl n} (k+g-1) }$.
\end {block}
\end {frame}

\begin {frame} {Условие применения декодера. Случай 1.}
\begin {block} {Утверждение 1.1}
Пусть $\sqrt { {\sl n} (k+g-1) } \notin {\mathbb N}$, тогда максимальный возможный радиус декодера Судана-Гурусами равен ${\sl r_*}  =  {\sl n} - \lceil \sqrt { {\sl n} (k+g-1) } \rceil$.
\end {block}

\begin {block} {Утверждение 1.2}
Пусть ${\sl n}$  - длина кода,  ${\sl k}$  - размерность кода, ${\sl g}$ - род кривой, на которой определён код, ${\sl r}$ - радиус списочного декодера, $D$ - дивизор кода $C$, ${\rm deg}(D) = \alpha \ge 2g - 1.$ Пусть $\sqrt { {\sl n}\alpha } \notin {\mathbb N}$.
Тогда списочный декодер Судана-Гурусвами для алгеброгеометрического кода {\sl C} применим в ССШШ, если:
\begin{equation}
 {\sl c} < \frac {\sl n}{\lceil \sqrt {{\sl n} (k+g-1)} \ \rceil}
\end {equation}

Причём при выполнении этого условия выполняется и условие (1) ($C$ является $c-TA$-кодом).
\end {block}
\end {frame}

\begin {frame} {Условие применения декодера. Случай 2.}
\begin {block} {Утверждение 2.1}
Пусть $\sqrt { {\sl n} (k+g-1) } \in {\mathbb N}$, тогда максимальный возможный радиус декодера Судана-Гурусами равен ${\sl r_*}  =  {\sl n} - \sqrt { {\sl n} (k+g-1) } - 1$.
\end {block}

\begin {block} {Утверждение 2.2}
Пусть ${\sl n}$  - длина кода,  ${\sl k}$  - размерность кода, ${\sl g}$ - род кривой, на котором определён код, ${\sl r}$ - радиус списочного декодера, $D$ - дивизор кода $C$, ${\rm deg}(D) = \alpha \ge 2g - 1.$  Пусть $\sqrt { {\sl n}\alpha } \in {\mathbb N}$.
Тогда списочный декодер Судана-Гурусвами для алгебро-геометрического кода {\sl C} применим в ССШШ, если:
\begin{equation}
 {\sl c} < \frac {\sl n}{\sqrt {{\sl n} (k+g-1)} + 1}
\end {equation}

Причём при выполнении этого условия выполняется и условие (1) ($C$ является $c-TA$-кодом).
\end {block}
\end {frame}

\begin {frame} {Двойственное условие на род кривой}
\begin {block} {Утверждение 3.1}
Пусть  $C$ --- алгеброгеометричсекий код, ${\sl n}$  --- длина кода,  ${\sl k}$  --- размерность кода, ${\sl g}$ --- род кривой, на котором определён код, $D$ --- дивизор кода $C$, ${\rm deg}(D) = \alpha \ge 2g - 1$.

Если
\begin {equation}
 {\sl g} < 1 - k + \frac {\sl n}{\sl c^2},
\end {equation}
то выполняется условие (1) ($C$ является $c-TA$-кодом).
\end {block}
\end {frame}

\begin {frame} {Двойственное условие на род кривой. Случай 1}
\begin {block} {Утверждение 3.2}
Пусть $C$ --- алгеброгеометричсекий код, ${\sl n}$  --- длина кода,  ${\sl k}$  --- размерность кода, ${\sl g}$ --- род кривой, на котором определён код, $D$ --- дивизор кода $C$, ${\rm deg}(D) = \alpha \ge 2g - 1$.  Пусть $\sqrt { {\sl n}\alpha } \notin {\mathbb N}$. Обозначим $ \epsilon =\lceil  \sqrt {{\sl n} \alpha} \ \rceil - \sqrt {{\sl n} \alpha} $.

Если
\begin {equation}
 {\sl g} < 1 - k + \frac {\sl n}{\sl c^2} - \frac {2\epsilon}{\sl c} + \frac {\epsilon ^2}{\sl n},
\end {equation}
то выполняется условие (2) (условие применение декодера Судана-Гурусвами).

Причём при выполнении (5) выполняется и (4).
\end {block}
\end {frame}

\begin {frame} {Двойственное условие на род кривой. Случай 2}
\begin {block} {Утверждение 3.3}
Пусть  $C$ --- алгеброгеометричсекий код, ${\sl n}$  --- длина кода,  ${\sl k}$  --- размерность кода, ${\sl g}$ --- род кривой, на котором определён код, ${\sl r}$ --- радиус списочного декодера,  $D$ --- дивизор кода $C$, ${\rm deg}(D) = \alpha \ge 2g-1$. Пусть $\sqrt { {\sl n}\alpha } \in {\mathbb N}$.
Если
\begin {equation}
 {\sl g} < 1 - k + \frac {\sl n}{\sl c^2} - \frac {2}{\sl c} + \frac {1}{\sl n},
\end {equation}
то выполняется (3) (условие применение декодера Судана-Гурусвами).

Причём, если выполняется (6), то выполняется и (5),
а значит, и (1)  ($C$ является $c-TA$-кодом).

\end {block}
\end {frame}

\section {Алгеброгеометрические коды ($\Omega$ - конструкция)}

\begin {frame}
\tableofcontents[currentsection]
\end {frame}

\begin {frame} {Условие c-TA}
 $\Omega$ - конструкция алгеброгеометрических кодов --- двойственные по отошению к $L$-конструкции коды. Будем обозначать алгеброгеометрический код ${\Omega}$-конструкции $C_{\Omega}$, его длину --- $n_{\Omega}$, размерность  $k_{\Omega}$.

\begin {block} {Условие c-TA}
Пусть $C_{\Omega}$ определён на кривой рода $g$, $D$ --- дивизор кода, ${\rm deg}(D) \ge 2g - 1$, тогда $C_{\Omega}$  является c-TA кодом, если выполнено условие:
$$ {\sl c} < \sqrt {\frac {\sl n_{\Omega}}{k_{\Omega}+g-1}} .$$
\end {block}

Проблема использования в отсутствии списочного декодера.
\end {frame}

\begin {frame} {Классические коды Гоппа}
Классические коды Гоппа (в том числе бинарные коды Гоппа) --- класс алгеброгеометрических кодов ${\Omega}$-конструкции.

Бинарный код Гоппа $g$ зависит от трёх параметров: $g = g(n, m, t)$.

\begin {block} {Условие c-TA}
Бинарный код Гоппа $g(n,m,t)$ является c-TA кодом, если выполнено условие:
$$ {\sl c} < \sqrt {\frac {\sl n}{n - 2t - 1}}.$$
\end {block}
\end {frame}

\begin {frame} {Списочный декодер Бернштейна}
Списочный декодер Бернштейна --- списочный декодер для бинарных кодов Гоппы.
\newline

Декодер исправляет $\lfloor n - \sqrt {n(n-2t-2)} \rfloor$ ошибок.

\begin {block} {Условие применения декодера}
Списочный декодер Бернштейна применим в ССШШ, если выполнено условие:
$$ {\sl c} < \frac {\sl n}{\lceil \sqrt{n(n - 2t - 1)} \rceil}.$$
\end {block}
\end {frame}

\begin {frame}
Спасибо за внимание!
\end {frame}
\end {document} 